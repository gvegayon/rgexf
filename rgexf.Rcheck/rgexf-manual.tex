\nonstopmode{}
\documentclass[letterpaper]{book}
\usepackage[times,hyper]{Rd}
\usepackage{makeidx}
\usepackage[utf8,latin1]{inputenc}
\makeindex{}
\begin{document}
\chapter*{}
\begin{center}
{\textbf{\huge Package `rgexf'}}
\par\bigskip{\large \today}
\end{center}
\begin{description}
\raggedright{}
\item[Type]\AsIs{Package}
\item[Title]\AsIs{An R library to build GEXF network files}
\item[Version]\AsIs{1.0}
\item[Date]\AsIs{2012-02-28}
\item[Author]\AsIs{George Vega Yon }\email{george.vega@nodoschile.org}\AsIs{}
\item[Maintainer]\AsIs{George Vega Yon }\email{george.vega@nodoschile.org}\AsIs{}
\item[Description]\AsIs{Using data frames with nodes, edges, attributes and
date/time data, builds GEXF files.}
\item[License]\AsIs{GPL >= 2}
\item[LazyLoad]\AsIs{yes}
\end{description}
\Rdcontents{\R{} topics documented:}
\inputencoding{utf8}
\HeaderA{rgexf-package}{An R library tu build GEXF graph files}{rgexf.Rdash.package}
\aliasA{rgexf}{rgexf-package}{rgexf}
%
\begin{Description}\relax
Builds GEXF files
\end{Description}
%
\begin{Details}\relax

\Tabular{ll}{
Package: & rgexf\\{}
Type: & Package\\{}
Version: & 1.0\\{}
Date: & 2012-02-28\\{}
License: & What license is it under?\\{}
LazyLoad: & yes\\{}
}
\end{Details}
%
\begin{Author}\relax
George Vega Yon <george.vega@nodoschile.org>

Maintainer: George Vega Yon <george.vega@nodoschile.org>
\end{Author}
\inputencoding{utf8}
\HeaderA{addEdges}{Function to add edges to the GEXF object}{addEdges}
%
\begin{Description}\relax
Add edges
\end{Description}
%
\begin{Usage}
\begin{verbatim}
addEdges(x, parent, att = NA, life = NA, ...)
\end{verbatim}
\end{Usage}
%
\begin{Arguments}
\begin{ldescription}
\item[\code{x}] A two (three) column matrix containing source, target ids
\item[\code{parent}] xml Parent
\item[\code{att}] A list of attributes of the same length of x
\item[\code{life}] A list or a two column matrix with start and end
\item[\code{...}] Arguments to pass on newXMLnode.
\end{ldescription}
\end{Arguments}
%
\begin{Author}\relax
George Vega Yon <george.vega@nodoschile.org>
\end{Author}
\inputencoding{utf8}
\HeaderA{addNodes}{Function to add nodes to the GEXF object}{addNodes}
%
\begin{Usage}
\begin{verbatim}
addNodes(x, parent, att = NA, life = NA, ...)
\end{verbatim}
\end{Usage}
%
\begin{Arguments}
\begin{ldescription}
\item[\code{x}] A two (three) column matrix containing id, label
\item[\code{parent}] xml Parent
\item[\code{att}] A list of attributes of the same length of x
\item[\code{life}] A list or a two column matrix with start and end
\item[\code{...}] Arguments to pass on newXMLnode.
\end{ldescription}
\end{Arguments}
%
\begin{Author}\relax
George Vega Yon <george.vega@nodoschile.org>
\end{Author}
\inputencoding{utf8}
\HeaderA{checkTimes}{Checks for correct time format}{checkTimes}
%
\begin{Description}\relax
Checks time
\end{Description}
%
\begin{Usage}
\begin{verbatim}
checkTimes(x, format = "date")
\end{verbatim}
\end{Usage}
%
\begin{Arguments}
\begin{ldescription}
\item[\code{x}] A string char
\item[\code{format}] format, could be "date", "dateTime", "float"
\end{ldescription}
\end{Arguments}
%
\begin{Author}\relax
George Vega Yon <george.vega@nodoschile.org>
\end{Author}
\inputencoding{utf8}
\HeaderA{defEdgesAtt}{Defines edges att}{defEdgesAtt}
%
\begin{Description}\relax
Internal
\end{Description}
%
\begin{Usage}
\begin{verbatim}
defEdgesAtt(...)
\end{verbatim}
\end{Usage}
%
\begin{Arguments}
\begin{ldescription}
\item[\code{...}] Further arguments
\end{ldescription}
\end{Arguments}
%
\begin{Author}\relax
George Vega Yon <george.vega@nodoschile.org>
\end{Author}
\inputencoding{utf8}
\HeaderA{defNodesAtt}{Add nodes attributes}{defNodesAtt}
%
\begin{Description}\relax
Internal
\end{Description}
%
\begin{Usage}
\begin{verbatim}
defNodesAtt(x, parent, ...)
\end{verbatim}
\end{Usage}
%
\begin{Arguments}
\begin{ldescription}
\item[\code{x}] List of attributes
\item[\code{parent}] xml parent
\item[\code{...}] Further arguments to pass on to newXMLNode
\end{ldescription}
\end{Arguments}
%
\begin{Author}\relax
George Vega Yon <george.vega@nodoschile.org>
\end{Author}
\inputencoding{utf8}
\HeaderA{gexf}{Exports to GEXF}{gexf}
%
\begin{Description}\relax
Based on nodes and edges data frames builds XML format GEXF file
\end{Description}
%
\begin{Usage}
\begin{verbatim}
gexf(nodes, edges, edgesAtt = NA, nodesAtt = NA, nodeDynamic = NA, output = NA, tFormat = "double", defaultedgetype = "undirected")
\end{verbatim}
\end{Usage}
%
\begin{Arguments}
\begin{ldescription}
\item[\code{nodes}] Two columns data frame including 'label' and 'id'
\item[\code{edges}] Two columns data frame including 'source' and 'target' based on nodes ids
\item[\code{edgesAtt}] na
\item[\code{nodesAtt}] na
\item[\code{nodeDynamic}] Two columns data frame including 'start' and 'end' according to the format specified at tFormat
\item[\code{output}] Complet path (including file name) where to print the GEXF file
\item[\code{tFormat}] Time format for dynamic graphs
\item[\code{defaultedgetype}] 'directed' or 'undirected'
\end{ldescription}
\end{Arguments}
%
\begin{Author}\relax
George Vega Yon <george.vega@nodoschile.org>
\end{Author}
\inputencoding{utf8}
\HeaderA{print.gexf}{Printing method for gexf class}{print.gexf}
%
\begin{Description}\relax
To print graph gexf objects on screen or into a file
\end{Description}
%
\begin{Usage}
\begin{verbatim}
print.gexf(x, file, replace = F, ...)
\end{verbatim}
\end{Usage}
%
\begin{Arguments}
\begin{ldescription}
\item[\code{x}] A gexf class object (builded from gexf())
\item[\code{file}] Complete output path
\item[\code{replace}] Logical. if TRUE replaces the target file (if exists)
\item[\code{...}] Further arguments to pass on print
\end{ldescription}
\end{Arguments}
%
\begin{Author}\relax
George Vega Yon <george.vega@nodoschile.org>
\end{Author}
\printindex{}
\end{document}
